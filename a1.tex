
\documentclass{article}
\usepackage{fullpage}
\usepackage{graphicx}
\usepackage{listings}
\usepackage{tabularx}
\begin{document}
\section{Part A: Tracking Process Activity}
\subsection{Hardware Information}


\begin{center}
\begin{tabular}{|l|l|}
  	\hline
  	CPU name & Intel Pentium G630 (Dual core)\\ \hline
  	CPU speed & 2ls.70GHz \\ \hline
  	L1 Cache Size & 32 KB \\ \hline
	L2 Cache Size & 256 KB \\ \hline
	L2 Cache Size & 6144 KB \\ \hline
	Memory & 7874 MB \\ \hline
\end{tabular}
\end{center}


\section{Part B: Measuring Numa Effects}
\subsection{Hardware Information}

This experiement was done on the wolf server.

\begin{center}
\begin{tabular}{|l|l|}
  	\hline
  	CPU name & AMD Opteron Processor 6348\\ \hline
  	CPU speed & 2.80GHz \\ \hline
  	L1 Cache Size & 64 KB \\ \hline
	L2 Cache Size & 2048 KB \\ \hline
	L3 Cache Size & 6144 KB \\ \hline
	Memory & 64408 MB \\ \hline
\end{tabular}
\end{center}

\subsection{Initialization}

Using the \"numactl --hardware\" command, I found that the machine (wolf.cdf) contains 4 nodes 

These nodes have about 12 CPU cores each, represented by the data collected below:

\begin{center}
\begin{tabular}{|l|l|}
  \hline
  Node & CPU \\ \hline
  0 & 0, 1, 2, 3, 4, 5, 6, 7, 8, 9, 10, 11 \\ \hline
  2 & 12, 13, 14, 15, 16, 17, 18, 19, 20, 21, 22, 23 \\ \hline
  4 & 24, 25, 26, 27, 28, 29, 30 ,31, 32, 33, 34, 35 \\ \hline
  6 & 36, 37, 38, 39, 40, 41, 42, 43, 44, 45, 46, 47 \\ \hline
\end{tabular}
\end{center}

With Node distances:

\begin{center}
\begin{tabular}{|l|l|l|l|l|}
	\hline
	Node Distances \\ \hline
	Node & 0 & 2 & 4 & 6 \\ \hline
  	0 & 10 & 16 & 16 & 16 \\ \hline
  	2 & 16 & 10 & 16 & 16 \\ \hline
  	4 & 16 & 16 & 10 & 16 \\ \hline
  	6 & 16 & 16 & 16 & 10 \\ \hline

\end{tabular}
\end{center}



\subsection{Test Script}

The results of our script can be found in my\_part\_b\_results.txt and the script itself is run\_experiment\_B

\subsection{Observations}

From my results file mentioned above by looking at some of the operations from the CPUs below:

\begin{center}
\begin{tabular}{|l|l|l|l|l|}
	\hline 
	CPU & Copy & Scale & Add & Triad \\ \hline
	0 & 5842.4 & 5819.0 & 6559.7 & 6540.0 \\ \hline
	2 & 5853.9 & 5785.0 & 6429.5 & 6538.6 \\ \hline
	14 & 2970.5 & 2970.8 & 3014.5 & 3014.3 \\ \hline
	15 & 2974.2 & 2971.6 & 3017.1 & 3017.2 \\ \hline
\end{tabular}
\end{center}

\end{document}